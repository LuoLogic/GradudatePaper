\chapter{总结和展望}
\thispagestyle{others}
\pagestyle{others}
\xiaosi

\section{主要结论}
近些年,得益于深度学习理论的不断发展,基于深度学习的计算机视觉也在各行各业得到的前所未有的广泛应用。然而,截止目前,随着相关深度学习模型结构的愈发复杂,参数量的日益庞大,使得研究人员对深度神经网络的可靠性也提出了新的要求。在一些关键场景下,深度神经网络的应用仍然受限。究其原因,深度神经网络的仍然存在诸多的可靠性问题,无法完全应对处理现实世界中的所有情况,可能会在一些意想不到的场景下出现失误,这对诸如自动驾驶,人脸识别,航空航天等高可靠性领域而言是不可接受的。因此,深度学习的可解释研究近年来逐渐兴起。在基于深度学习的图像分类领域,即便目前的众多模型在各种数据集的分类任务上表现出了优越的性能,但是研究者仍然希望得到深度神经网络的分类依据,即在输入图片中找到图像分类模型决策所依赖的关键特征。所以,近些年诞生了多种针对图像分类神经网络的视觉解释技术,它们依靠生成显著图的方式来给予输入图片中每个像素一个权值来表达其对输出结果分类的贡献值。本研究就是在这种背景下以图像分类神经网络的显著图作为研究对象,分析当前显著图生成技术存在的弊端,从而更好地生成符合图像分类神经网络决策的显著图。本研究的主要工作和结论如下:

(1)本研究提出了一种新的针对基于卷积神经网络的图像分类模型的决策结果生成高分辨率的显著图的方法。该方法采用多尺度放大原始输入图像,并将其输入到图像分类模型中。通过从最后一层卷积层获取不同分辨率的特征图集合,并对目标类别的分数进行反向传播,得到最后一层卷积层特征图对应的梯度矩阵。接着,将不同分辨率的特征图集合和梯度矩阵集合融合为原始输入图像的分辨率,并进行加权相加以生成掩膜。将所有掩膜分别应用于扰动原始输入图像,并输入到图像分类模型中,以获取目标类别的概率分数作为掩膜的权重。最后,对所有掩膜和其对应的概率分数进行加权相乘,生成最终的显著图。本研究以在ILSVRC 2012数据集、PASCAL VOC数据和COCO2014数据集这几个公开的数据集进行了直观评估和数据实验,直观评估的结果显示该方法生成的显著图相比目前著名的其他类型方法具有更加细粒度的可视化效果,分辨率更高,提供更直观的可视化解释结果。数据实验的结果显示该方法相比其他对比方法能够更准确地揭示图像分类神经网络在输入图像中的决策依据。

(2)本研究提出了一种通用的显著图增强方法,旨在解决当前显著图解释方法生成低分辨率显著图的问题。该方法适用于多种显著图解释方法,可以在不改变原有方法内部计算流程的情况下,达到提高显著图分辨率,增强显著图对特征的定位能力的效果。该方法通过使用固定尺寸的滑动窗口对输入图像的局部区域进行上采样,并将结果输入到选定的显著图解释方法中,得到针对特定类别的显著图和概率分数。随后,将显著图下采样到输入图像对应位置的窗口中,并乘以概率分数,生成具备更多细节高分辨率的显著图。将该方法在不同的显著图生成方法上的应用结果表明,在直观评估对比中,该增强方法能够显著图提升对比方法的显著图生成效果,相比对比方法能提供更加细粒度的视觉效果。在数据实验中,该增强方法能够显著提升对比方法的扰动实验的指标数据,这表明该增强方法可以增强对比方法的显著图解释效果,提供更符合图像分类模型决策结果的显著图。在分割实验中,该增强方法也展现出了更强的目标类别图像分割能力。

\section{研究展望}
随着深度学习理论的继续发展以及深度神经网络在社会智能化趋势中的广泛应用,深度学习的可解释性研究也将逐渐成为不可或缺的一个重要研究分支。那么在计算视觉领域,基于图像分类神经网络的显著图相关研究也将会得到更加广泛的关注,它不仅能够提供视觉解释,还能够为研究者帮助训练神经网络,为研究者判断神经网络是否真正学习到了重要特征提供判别依据。未来的研究中,对图像分类神经网络的显著图生成研究可能有以下几个方向的发展。

首先是对生成高分辨率显著图的研究。目前绝大多数显著图生成方法都存在生成的显著图分辨过低导致无法精确锁定重要特征的问题。基于类激活映射的方法受限于特征图分辨率,而基于扰动的方法则受限于扰动掩膜的分辨率和计算量的限制,基于反向归因传播的方法在某些模型下虽然能生成较高分辨率的显著图,但是显著图视觉效果较为离散,且噪点过多。因此未来对于高分辨率显著图的研究会更加受到重视。

其次是对新型图像分类模型的研究。近些年,基于Transformer架构的图像分类模型以其优越的性能表现迅速得到了广泛研究,但是针对基于Transformer架构的图像分类模型的显著图生成方法却寥寥无几,绝大多数方法仍然只能应用在基于卷积神经网络的模型上,即便有部分研究者提出了针对基于Transformer架构显著图生成方法,其视觉解释效果也不够好,因此这将是未来一个值得重点研究的新方向。

最后是探索完全符合图像分类神经网络决策的显著图生成方法。目前的显著图生成方法给出的显著图虽然能够大致反映图像分类模型的感兴趣区域和特征,但是仍然存在诸多问题。例如显著图对每个像素给出权值并不是完全对应于其对输出结果的贡献,该权值不能完全单一的看待,只能和周围区域的像素权值一起看待来大概反映该片区域对图像分类模型输出结果的贡献。因此一种完全符合图像分类神经网络决策的显著图生成方法可以值得在未来进行深入研究。