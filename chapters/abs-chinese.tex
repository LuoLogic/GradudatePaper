%中文摘要,自行编辑内容



\chapter{摘\quad 要}
\xiaosi

当下深度学习技术在众多领域都展现了其应用潜力,在图像分类领域,随着模型结构的日益复杂,参数量逐渐提升,人们很难对图像分类模型给出的分类结果给出可靠的解释。显著图解释技术作为深度学习可解释工作的一个新兴分支,其研究目的是从输入图片中找到图像分类模型做出决策的依据成从而给出可视化的解释。但是当前的显著图解释技术普遍存在生成的原始显著图分辨率较低,无法精确定位目标关键特征等缺陷。因此本研究针对当前显著图解释技术存在的缺陷入手,研究如何提高显著图的分辨率,增加显著图的呈现的有效信息,提高显著图对图像分类神经网络决策结果的可视化解释能力。本研究的主要工作如下:

首先,本研究提出了一种针对基于卷积神经网络的图像分类模型的显著图解释方法。该方法通过将原始输入图片进行多尺度的放大并分别输入到图像分类模型当中,从图像分类神经网络的最后一层卷积层获取得到不同分辨率的特征图集合。再分别对目标类别的分数进行反向传播得到最后一层卷积层特征图对应的梯度矩阵。然后将不同分辨率特征图和梯度矩阵融合为和原始输入图片一样的分辨率并加权相加得到掩膜。将所有掩膜分别扰动原始输入图片后输入至图像分类模型当中得到目标类别的概率分数作为掩膜的权重。最后将所有掩膜和其对应的概率分数加权相乘得到最终的显著图。在三个公开的数据集( ILSVRC 2012数据集,PASCAL VOC数据集和COCO2014数据集)上进行定性和定量实验,该方法生成的显著图具有更高分辨率,能够更加精准地找到图像分类神经网络在输入图片中的决策依据,提供更加直观的可视化解释结果。

其次,针对目前显著图解释方法生成的原始显著图普遍低分辨率的情况,本研究提出了一种通用的显著图增强方法,可以直接应用在多数显著图解释方法上。该方法使用固定尺寸的滑动窗口对输入图片中的所有局部区域上采样到输入图片尺寸,然后将结果输入到选定的显著图解释方法中得到所有图片的针对特定类别的显著图和概率分数,最后将显著图下采样到输入图片对应位置上的窗口中,并乘以概率分数,即可得到具备更多细节的显著图。本研究将该方法应用在不同的显著图解释方法上,无论是量化指标还是直观评测都显示出该方法明显提升了其他显著图解释方法生成显著图的质量,从而证明该方法的有效性和可靠性。

最后,针对目前众多显著图生成方法实现复杂,难以直观对比不同方法生成的显著图质量优劣的问题,本研究设计并实现了一个显著图解释方法对比评测系统。
\\

\noindent\songti\textbf{关键词:}图像分类神经网络,深度学习可解释性,显著图,类激活映射

\clearpage