


\chapter{相关理论介绍
}
\thispagestyle{others}
\pagestyle{others}
\xiaosi

\section{引言}
解释深度神经网络引起了越来越多的关注,因为它有助于理解网络的内部机制以及网络做出特定决策的原因。在计算机视觉领域中,可视化和理解深度神经网络最流行的方法之一是生成与网络决策相关的显著区域的显著性图。许多深度神经网络相关的可解释方面的研究和方法都可以在图像分类神经网络上生成显著图。显著图生成的质量可以直观反映不同可解释或者可视化算法的优劣,此外显著图还可以作为图像弱监督分割和目标的定位的一种手段,因其可以反映目标物体在图像中的空间位置而且其只需要训练好的图像分类神经网络即可完成任务。


%一些早期的研究单纯通过反向传播的梯度差异来生成显著图描述图像分类模型在输入图片中感兴趣的区域,后来
深度神经网络的可解释方面的研究是在最近十年才逐渐兴起并收到关注的,在计算机视觉领域,基于深度卷积神经网络的图像分类模型是较早受到研究的,研究者试图从参数量庞大的深度卷积神经网络中找到输出结果和在输入图片中对应的依据。也有一些研究者将图像分类神经网络看作是一个黑盒,通过各种手段扰动输入图片观测输出结果变化来生成显著图。随着Transformer架构异军突起,基于Transformer架构的图像分类神经网络的可解释性也逐渐受到关注和研究,也已经由研究者设计了针对Transformer架构的反向传播归因机制,该机制在计算机视觉领域也能生成效果良好的显著图。上述的深度神经网络可解释研究生成的显著图较少关注显著图生成的质量和对关键特征的定位能力,本文的显著图解释研究专注于对显著图生成质量的改善和相关显著图解释算法的改善。

接下来本章首先介绍图像分类神经网络的主流架构概念包括基于卷积神经网络(CNN)的和基于Transformer架构的,然后介绍三种著名的深度学习可解释算法,这些算法在图像分类神经网络上也能生成显著图,最后介绍当前显著图解释算法的评价指标。

\section{卷积神经网络 }

\section{}

学位论文包括前置部分、主体部分和结尾部分共三大部分,各部分组成及顺序如所示。

学位论文各部分独立为一部分,每部分应从新的一页开始。

论文的正文(中间各章)是论文的核心部分,一般由标题、文字叙述、图、表格和公式等部分构成。由于涉及的学科、选题、研究方法等有很大的差异,可以有不同的写作表达方式,但应遵循本学科通行的学术规范,必须实事求是,客观真切,准确完备,合乎逻辑,层次分明,简练可读。引用他人研究成果时,应注明出处,不得将其与本人的工作混淆。


\section{字数要求}
字数要求


\subsection{硕士论文要求}

各学科和学部自定。

\subsection{博士论文要求}

各学科和学部自定。

\section{字体和段落}
学位论文中的中文统一用宋体,数字和英文统一用Times New Roman字体。从中文摘要开始,所有文字段落和标题行间距均取固定值20磅;所有段落按两端对齐、首行缩进2个全角字符方式书写内容。

中、英文混排时,除小数点以及引用的分图序号、公式序号等外,宜使用全角标点符号(逗号、冒号、括号、引号等);英文段落中,符号使用应遵循英文书写习惯,统一使用半角符号,并规范使用空格。

其他要求:

(1)各级标题不得置于页面的最后一行,即须与下段同页;

(2)两个标题之间无正文时,第二个标题的段前距设置为0磅;

(3)图、表、公式统一采用单倍行距;

(4)只有一、两行文字的,不得单独作为一页内容;除各章最后一页外,中间页面不得出现较大空白;

(5)必要时,可在规定的格式要求基础上适当微调,以利于排版,但显示效果不得与规定的格式要求存在明显差距。

%调整图片与上方文字之间的间距
\vspace{-0.15cm}

\begin{figure}[h]
	\centering 
	\includegraphics[width=10cm]{chapters/31}
	\bicaption[\xiaosi 不同章节图片排版测试]{\wuhao 图片排版测试}{\wuhao Scaling results with different scaling coefficients ν}
	\label{fig:2.1}
\end{figure}

%调整图片与下方文字之间的间距
\vspace{-0.5cm}

\section{章节标题}
目录中章节标题只显示到3级标题,正文中最多显示到4级标题。

\subsection{三级标题}

\vspace{0.5cm}

\subsubsection{四级标题}

\vspace{0.3cm}



\section{本章小结}
本章介绍了……




