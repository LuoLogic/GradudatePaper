% 致谢
%\specialsectioning
\chapter{致 \quad 谢}
\thispagestyle{others}
\pagestyle{others}
\xiaosi

不知不觉间,已经快过了三年了,研究生生涯也即将画上句号,或者说在校学习的生活也会画上一个句号。在即将步入社会的这个时间点,回望我整个学生时期,有太多的话想要一一写下来,但是,最值得写在这里的就是对在我人生成长道路上帮助我,一起同行的人的感谢。没有他们的帮助陪伴,我不会走到今天。我很幸运,在自己的人生道路上碰到了如此多的良师益友。

首先,我要感谢我学习生涯中遇到的所有老师们。我很幸运我自己从小学到研究生期间遇到的老师都很好,他们不仅传授了知识,也是对我的品格养成产生了重要影响。这里感谢重庆邮电大学的项小红老师和邓欣老师,他们治学严谨,和蔼可亲,在我研究生阶段给予了悉心的帮助和指导。感谢南京林业大学的杨绪兵教授和张礼老师,他们在我本科阶段给予了很大的帮助。还要感谢眉山市第一中学的汤琪君和夏春雨等老师,苏祠中学的唐剑忠和张强等老师,苏州市黄埭中心小学的钱雪芬等老师,他们在我心中都是非常好的老师,他们尽心尽责、关爱学生、幽默风趣、循循善诱,在我心中留下了深刻的烙印。

其次,我要感谢这一路陪伴我同行的同学和朋友们。他们在我人生的道路上也给予了莫大的帮助,让我一路不在孤独。感谢陈泽瑜同学在研究生阶段陪伴我给予我鼓励支持。还要感谢中国科学院信息工程研究所的在读博士生魏成安同学。他是我的高中挚友,他在我的研究生期间的科研工作中给予了非常大的帮助。我一直记得暑假的很多个夜晚,我和他一起在眉山城区街头散步,谈天说地的时光。这里祝他毕业顺利。

同时,我要感谢我的父母。他们勤劳善良,热爱生活,辛勤抚育我长大,他们对我的恩情我无以为报,只盼望能在工作后好好回报他们。

最后,感谢评审专家和答辩专家们,您们辛苦了。




