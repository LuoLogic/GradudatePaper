\specialsectioning


\chapter{作者简介}
\thispagestyle{others}
\pagestyle{others}
\xiaosi

\section{1. \ 基本情况}
张涪源,男,四川人,1998年10月出生,重庆邮电大学计算机科学与技术学院计算机技术专业2021级硕士研究生。

\section{2. \ 教育和工作经历}



2017.09~2021.06 南京林业大学信息科学技术学院,本科,专业:软件工程

2021.09~2024.09 重庆邮电大学计算机科学与技术学院,硕士研究生,专业:计算机技术

\section{3. \ 攻读学位期间的研究成果}

\subsection{3.1 \ 发表的学术论文和著作}
\hangafter 1
\hangindent 1.5em
\noindent
[1] \textbf{Zhang F}, Xiang X, Deng X, et al. Multi-size Scaled CAM for More Accurate Visual Interpretation of CNNs[C]//International Conference on Neural Computing for Advanced Applications. Singapore: Springer Nature Singapore, 2023: 147-161.(EI会议,已发表)

\hangafter 1
\hangindent 1.5em
\noindent
[2] Xiang X, \textbf{Zhang F}, Deng X, et al. MSG-CAM: Multi-scale inputs make a better visual interpretation of CNN networks[C]//2023 IEEE International Conference on Multimedia and Expo (ICME). IEEE, 2023: 312-317.(CCF-B类会议,已发表)

\hangafter 1
\hangindent 1.5em
\noindent
[3] Xiang X, \textbf{Zhang F}, Deng X, et al. Visualization Enhancement of Saliency Methods Based on the Sliding Window Mechanism[M]//ECAI 2023. IOS Press, 2023: 2728-2735.(CCF-B类会议,已发表)

\subsection{3.2 \ 申请(授权)专利}

\hangafter 1
\hangindent 1.5em
\noindent
[1]项小红,\textbf{张涪源},邓欣,丁晓宇.一种基于上采样机制和类激活映射的图像分类结果特征可视化方法.中国,CN202310400157.2[P].2023.04.14.(已受理)

\hangafter 1
\hangindent 1.5em
\noindent
[2] 项小红,\textbf{张涪源},邓欣,张浩.一种基于滑动窗口机制的图像分类神经网络可视化算法的增强方法.中国,CN202310428053.2[P].2023.07.07.(已受理)


\subsection{3.3 \ 参与的科研项目及获奖}






